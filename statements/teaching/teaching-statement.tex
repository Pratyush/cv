\synctex=1


\documentclass[11pt,letterpaper]{article}
\usepackage[notes=true,later=false]{../dtrt}
\usepackage[utf8]{inputenc}
\usepackage[margin=1in]{geometry}

\usepackage{amsmath}
\usepackage{amsthm}
\usepackage{mathtools}
\usepackage{amsfonts}
\usepackage{amssymb}
\usepackage{microtype}
\usepackage{newtxtext}
\usepackage{xspace}
\usepackage[inline,shortlabels]{enumitem}
  \setlist[itemize]{leftmargin=*}
  \setlist[enumerate]{leftmargin=*}
  \setlist[description]{leftmargin=*}
\usepackage{hypcap}

\usepackage{tabu}
\usepackage{hyperref}
\usepackage{booktabs}
\usepackage{float}
\usepackage{array}
\usepackage{color, soul}
\usepackage{subdepth}
\usepackage{varwidth}
\usepackage{xparse}
\usepackage{setspace}
\usepackage{titling}
\setlength{\droptitle}{-3cm}

%%%%%%%%%%%%%%%%%%%%%%%%%%%%%%%%%%%%%%%%%%%%%%%%%%%%%%%%%%%%%%%%%%%%%%%%%%%%%%%%
%%%%%%%%%%%%%%%%%%%%%%%%%%%%%%%%%%%%%%%%%%%%%%%%%%%%%%%%%%%%%%%%%%%%%%%%%%%%%%%%
%%%%%%%%%%%%%%%%%%%%%%%%%%%%%%%%%%%%%%%%%%%%%%%%%%%%%%%%%%%%%%%%%%%%%%%%%%%%%%%%

\newcommand{\arkworks}{\texttt{arkworks}}

\pagestyle{plain}

%%%%%%%%%%%%%%%%%%%%%%%%%%%%%%%%%%%%%%%%%%%%%%%%%%%%%%%%%%%%%%%%%%%%%%%%%%%%%%%%
%%%%%%%%%%%%%%%%%%%%%%%%%%%%%%%%%%%%%%%%%%%%%%%%%%%%%%%%%%%%%%%%%%%%%%%%%%%%%%%%
%%%%%%%%%%%%%%%%%%%%%%%%%%%%%%%%%%%%%%%%%%%%%%%%%%%%%%%%%%%%%%%%%%%%%%%%%%%%%%%%
\begin{document}

% Declarations for Front Matter

\title{Teaching Statement}
\author{Pratyush Mishra}
\date{}
%%%%%%%%%%%%%%%%%%%%%%%%%%%%%%%%%%%%%%%%%%%%%%%%%%%%%%%%%%%%%%%%%%%%%%%%%%%%%%%%

\maketitle
\vspace{-2em}

Applied cryptography has developed a plethora of exciting tools that enable seemingly impossible tasks: joint computations that hide all information about participants' inputs (MPC protocols), cryptographic proof systems that enable verifiers to succinctly check the validity of statement while learning nothing else (zkSNARKs), encrypted computations (homomorphic encryption), and more. These protocols have massive potential for improving the privacy, scalability, and integrity of existing systems, and the last decade has indeed seen some exciting (but limited) deployment of these. 

However, existing resources for learning such advanced applied cryptography are limited: security courses focus on simple cryptography (encryption, MACs, and hash functions), while cryptography courses focus on theoretical foundations (definitions and proofs).

My goal as a teacher is to fill this gap by \emph{democratizing cryptography}. I want to make applied cryptography accessible to, and exciting for, as wide an audience as possible. I intend to do this by creating and teaching undergraduate and graduate courses that train students in both the systems and theory parts of applied cryptography.

\parhead{My approach}
At a high level, my approach to teaching is to foster hands-on engagement with the course material.
As I demonstrate in my recap of my teaching experience below, I do so via two strategies.
First, I encourage active participation in lectures by having students participate in the process of developing cryptographic protocols.
This active engagement with the material helps solidify their understanding of the material, and also breaks the monotony of static lectures.
Second, to further concretize the abstract protocols we study in cryptography, I augment lectures with hands-on labs where the students and I jointly implement the protocols we learn about.


\parhead{Prior experience}
My goal is informed by my teaching experience. At Penn, I have taught two courses: CIS 5560, an advanced undergrad/masters level course on cryptography, and CIS 7000, a PhD-level seminar on cryptographic proof systems.
In both courses, I have tried to help students grasp abstract cryptographic concepts by grounding them in real-world examples.

The first course I taught at Penn was \textbf{CIS 7000: Theory and Practice of Succinct Zero Knowledge Proofs}, a PhD-level seminar on cryptographic proof systems.
The course had 12 students, with 4 undergraduates, 1 masters student, and 7 PhD students.
I designed this course from scratch in a hybrid format, with both lectures and paper presentations.
The lecture portions were a mix of theory and practice: after I introduced a new protocol, we would all complete an in-class lab where the students would implement the protocol.

This approach was well-received by students, and I received an instructor rating of $3.50/4.00$, and the overall course rating was $3.17/4.00$.

The next course I taught was \textbf{CIS 5560} (Cryptography), in \textbf{Spring 2024}. 
The course had 26 students, with 15 undergraduates and 11 masters students.
I developed the material by building upon resources from past iterations and similar courses at other institutions.
As it was my first time teaching this course, I did not deviate from the fairly standard syllabus (i.e., one-way functions, pseudorandomness, encryption, signatures, etc.) of these other courses.
As a result, the focus of the course was primarily theoretical, and assignments and assessments followed a traditional definition-and-proof format.
However, I did try to provide students with a feel of implementation aspects of cryptography by introducing an in-class lab where the students would collectively implement simple public-key encryption schemes.

I also tried to improve student engagement with theoretical components by avoiding a static lecture format and encouraging student participation.
For example, whenever introducing a new cryptographic construction, I built up to the final construction via a series of strawman constructions.
Starting from a naive construction, I asked students to identify flaws and suggest fixes, which would then lead to the next construction.
I believe this effort was noticed by students and improved their understanding of the material. Indeed, one student's feedback was as follows:
\begin{quotation}
  Student 1: \emph{One of the best professors I have had in the CIS department in terms of having active class participation. He took a math-heavy topic and really made it accessible to people of nay [sic] background.}
  
  Student 2: \emph{Prof. Mishra is really good at guiding students to think about cryptography concepts and able to lead high quality discussions during lectures \dots}
\end{quotation}

Overall, this was an exciting experience where I learnt how to teach undergraduates at Penn.
I received an instructor rating of $3.00/4.00$, and the overall course rating was $2.65/4.00$.

I am teaching \textbf{CIS 5560} again at the time of writing (\textbf{Spring 2025}).
This time around the course has 34 students, with 10 undergraduates and 24 masters students.
In this iteration, I have preserved and polished my high-engagement lecture style, and have attempted to improve upon other aspects by incorporating student feedback for the first iteration.

For example, a common feedback from students in the Spring 2024 iteration was that TAs would sometimes reschedule office hours abruptly, and would sometimes answer questions on Ed too late.
I have addressed these issues in the current iteration as follows.
I have instructed the TAs to avoid cancelling office hours, and instead coordinate to cover each other's office hours in case of emergencies.
Next, I have changed how homework design and question-answering responsibilities are assigned.
Now, each TA is responsible for the entire lifecycle of a homework assignment, from design, to question-answering, to grading.
As part of these responsibilities, the TA for a homework is expected to monitor Ed and answer student questions promptly.

I have also tried to improve the student community within the course by organizing weekly ``homework parties''. 
These are weekly gatherings where students can come together to work on homework assignments. 
All course staff are required to attend to help students with questions.
To incentivize student attendance and live up to the festive name, we provide pizzas to attendees.
These parties have been well-received, and attract a regular crowd of 10-15 students who have seemed to form a close-knit group.

\parhead{Future improvements and new courses}
I intend to build upon my experience so far to achieve my goal of democratizing cryptography as follows.
First, I intend to split up the current \textbf{CIS 5560} course into a two course sequence: 
\begin{enumerate*}[(a)]
  \item an upper-division undergraduate course that provides an intuitive introduction to the foundations of cryptography, but with less focus on definitions, theorems, and proofs, and more focus on implementation-based projects and assignments; and 
    
  \item a graduate course that provides a rigorous mathematical treatment of the foundations of cryptography and formalizes the concepts introduced in the undergraduate course.
\end{enumerate*}
In both courses, I plan to introduce not only classical cryptography (encryption, authentication, signatures, etc.), but also advanced applied cryptography (MPC, zero-knowledge proofs, etc.).
The implementation-oriented projects in the undergraduate course would exercise the theory taught in the course by requiring the students to build (simplified versions of) real-world systems that use cryptography to provide better privacy, scalability, and integrity guarantees.
\end{document}
